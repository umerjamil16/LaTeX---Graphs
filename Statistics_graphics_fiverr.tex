%%%%%%%%%%%%%%%%%%%%%%%%%%%%%%%%%%%%%%%%%%%%%%%%%%%%%%%%%%%%%%%%%%%%%%
%%%%%%%%%%%%%%%%%%%%%%%%%%%%%%%%%%%%%%%%%%%%%%%%%%%%%%%%%%%%%%%%%%%%%%
%%%%%%%%%%%%%%%%%%%%%%%%%%%%%%%%%%%%%%%%%%%%%%%%%%%%%%%%%%%%%%%%%%%%%%


% LexiCard, 03.05.2016

%%%%%%%%%%%%%%%%%%%%%%%%%%%%%%%%%%%%%%%%%%%%%%%%%%%%%%%%%%%%%%%%%%%%%%
%%%%%%%%%%%%%%%%%%%%%%%%%%%%%%%%%%%%%%%%%%%%%%%%%%%%%%%%%%%%%%%%%%%%%%
%%%%%%%%%%%%%%%%%%%%%%%%%%%%%%%%%%%%%%%%%%%%%%%%%%%%%%%%%%%%%%%%%%%%%%
% LexiCard, 05.02.2016
%%%%%%%%%%%%%%%%%%%%%%%%%%%%%%%%%%%
% LexiCard, 05.02.2016

\documentclass[final,hyperref={pdfpagelabels=false},xcolor=table]{beamer}
\mode<presentation>{
\usetheme{Statistik1}}
\setbeamersize{text margin left=0cm,text margin right=0cm}

%%%%%%%%%%%%%%%%%%%%%%%%%%%%%%%%%%%
\usepackage{grffile}
\usepackage{german} 
\usepackage[latin1]{inputenc} 
\usepackage[T1]{fontenc} 
\usepackage{amsmath,amsthm, amssymb, latexsym}
\usepackage[orientation=portrait,size=a4,scale=1,debug]{beamerposter}  
\usepackage{array,booktabs,tabularx}
\usepackage{etex} 
\usepackage{tikz}
\usepackage{wrapfig}
\usepackage{blindtext} 
\usepackage{multirow}
\usepackage{pgfplots}
\usepackage{colortbl} 
\usepackage{fancyvrb}
\usepackage[ISBN=978-80-85955-35-4]{ean13isbn} 		% Barcode 
%\usepackage[]{qrcode} 							% QR-Code 
\usepackage{calc}
\usepackage{graphicx}
\usepackage[absolute,overlay]{textpos}
\usepackage{anyfontsize}


\definecolor{spain}{RGB}{187,224,227}
\definecolor{uk}{RGB}{255,255,0}
\definecolor{france}{RGB}{255,128,128}
\definecolor{germany}{RGB}{0,255,0}
\definecolor{italy}{RGB}{153,153,255}
\definecolor{greece}{RGB}{255,204,0}
\definecolor{iceland}{RGB}{204,153,255}
\definecolor{ireland}{RGB}{0,255,255}
\definecolor{portugal}{RGB}{128,0,0}
\definecolor{spain}{RGB}{187,224,227}

%colors for figure9
\definecolor{gray1}{RGB}{166,166,166}
\definecolor{gray2}{RGB}{136,136,136}
\definecolor{green1}{RGB}{101,166,89}
\definecolor{green2}{RGB}{51,136,34}
\definecolor{orange1}{RGB}{255,191,101}
\definecolor{orange2}{RGB}{255,170,51}

\definecolor{findOptimalPartition}{HTML}{D7191C}
\definecolor{storeClusterComponent}{HTML}{FDAE61}
\definecolor{dbscan}{HTML}{ABDDA4}
\definecolor{constructCluster}{HTML}{2B83BA}
%%%%%%%%%%%%%%%%%%%%%%%%%%%%%%%%%%%
\usetikzlibrary{fadings}
\usetikzlibrary{shapes,arrows}
\usetikzlibrary{shapes.arrows} \usetikzlibrary{shadows}
\usetikzlibrary{tikzmark}
%\usetikzlibrary{paths.ortho} 
\usepgfplotslibrary{statistics}
\usetikzlibrary{intersections}
\usetikzlibrary{positioning}
\usetikzlibrary{patterns}
\usetikzlibrary{calc,shapes.multipart,arrows,decorations.pathmorphing,backgrounds,positioning,fit,petri}      

%%%%%%%%%%%%%%%%%%%%%%%%%%%%%%%%%%%
\tikzstyle{every picture}+=[remember picture]
\tikzset{shadowed/.style={preaction={transform canvas={shift={(1pt,-1pt)}},draw=Kasten1!80,line width = 3pt}},} \tikzset{samples=100}
\tikzset{declare function={gamma(\z)=
    (2.506628274631*sqrt(1/\z) + 0.20888568*(1/\z)^(1.5) + 0.00870357*(1/\z)^(2.5) - (174.2106599*(1/\z)^(3.5))/25920 - (715.6423511*(1/\z)^(4.5))/1244160)*exp((-ln(1/\z)-1)*\z);},
    declare function={gammapdf(\x,\k,\theta) = \x^(\k-1)*exp(-\x/\theta) / (\theta^\k*gamma(\k));}
}
\tikzset{declare function={student(\x,\n)= gamma((\n+1)/2.)/(sqrt(\n*pi) *gamma(\n/2.)) *((1+(\x*\x)/\n)^(-(\n+1)/2.));}}  
%%%%%%%%%%%%%%%%%%%%%%%%%%%%%%%%%%%
% Custom parameters
\setlength{\TPHorizModule}{10mm}
\setlength{\TPVertModule}{\TPHorizModule}
\textblockorigin{0mm}{0mm} % start everything near the top-left corner
\TPshowboxesfalse
\textblockcolour{} 
\TPMargin{1.5mm} 

\listfiles
\graphicspath{{fig/}}
 
\newlength{\columnheight}
\setlength{\columnheight}{28cm}


%%%%%%%%%%%%%%%%%%%%%%%%%%%%%%%%%%%
% Declare functions for pgf
\pgfmathdeclarefunction{gauss}{2}{%
  \pgfmathparse{1/(#2*sqrt(2*pi))*exp(-((x-#1)^2)/(2*#2^2))}%
}
%
\makeatletter
\pgfmathdeclarefunction{invgauss}{2}{%    
  \pgfmathln{#1}% <- might need parsing
  \pgfmathmultiply@{\pgfmathresult}{-2}%
  \pgfmathsqrt@{\pgfmathresult}%
  \let\@radius=\pgfmathresult%
  \pgfmathmultiply{6.28318531}{#2}% <- might need parsing
  \pgfmathdeg@{\pgfmathresult}%
  \pgfmathcos@{\pgfmathresult}%
  \pgfmathmultiply@{\pgfmathresult}{\@radius}%
}
%
\pgfmathdeclarefunction{randnormal}{0}{%
  \pgfmathrnd@
  \ifdim\pgfmathresult pt=0.0pt\relax%
    \def\pgfmathresult{0.00001}%
  \fi%
  \let\@tmp=\pgfmathresult%
  \pgfmathrnd@%
  \ifdim\pgfmathresult pt=0.0pt\relax%
    \def\pgfmathresult{0.00001}%
  \fi
  \pgfmathinvgauss@{\pgfmathresult}{\@tmp}%
}
\makeatother


%%%%%%%%%%%%%%%%%%%%%%%%%%%%%%%%%%%
% Custom commands
\makeatletter
\newcommand{\manuallabel}[2]{\def\@currentlabel{#2}\label{#1}}  
\makeatother

\newcommand{\pphantom}{\textcolor{red}} %
\newcommand\Warning{
 \makebox[1.4em][c]{%
 \makebox[0pt][c]{\raisebox{.1em}{\small!}}%
 \makebox[0pt][c]{\color{red}\Large$\bigtriangleup$}}}%




%%%%%%%%%%%%%%%%%%%%%%%%%%%%%%%%%%%
% Header even pages
\newcommand{\headereven}[1]{
\begin{tikzpicture}[remember picture,overlay]
\node [xshift=10.5cm,yshift=-1.5cm] at (current page.north west){
\begin{tikzpicture}
\shade[top color=Schattierung!80,bottom color=Hintergrund,shading angle=0]  (0,0) rectangle (21,3);
\end{tikzpicture}
};
\end{tikzpicture}


% Abschluss (Balken)
\begin{tikzpicture}[remember picture,overlay]
\node [xshift=9.8cm,yshift=-0.5cm] at (current page.north west){\begin{tikzpicture} % Achtung: Verschiebt den Mittelpunkt des Bildes, daher die seltsame Ma?angabe...
%\fill[white] (0,0) rectangle (21,0.05);
\draw[style=thick, white] (0,0) -- (19.1,0);
\draw[style=thick, white] (0,-0.05) -- (19.1,-0.05);
\end{tikzpicture}
};
\end{tikzpicture}



% Seitenzahlen
\begin{tikzpicture}[remember picture,overlay]
\node [xshift=9.8cm,yshift=-0.25cm] at (current page.north west){\begin{tikzpicture} 
\node (rect) at (4,2) [minimum width=0.5cm,minimum height=0.5cm,align=center,text=white] {-- #1 --};
\end{tikzpicture}};
\end{tikzpicture}

}

\usepackage{xcolor}
\usepackage{tikz}
\usepackage{pgfplots}
\usepackage{csvsimple}
\usepackage{pgf-pie}
\newlength{\xdim}
\usepackage{pgfplotstable}
%%for fig 7
\usepackage{filecontents}
%Data for Fig 7 [imp: c column scaled by 1000]
%% All the Y-axis values are scaled by 1000 to remove the void '10^4' 
\begin{filecontents*}{data.csv}
a,c
1, 11.43605 %eg. original value was 11436.05
2, 11.32880
3, 11.41962
4, 11.34060
5, 11.19715
8, 11.06492
9, 11.00129
10, 11.26539
11, 11.33278
12, 11.19649
15, 10.98497
16, 11.04401
17, 10.97801
18, 11.10030
19, 11.04010
22, 11.46050
23, 11.54254
24, 11.47126
25, 11.47313
26, 11.49243
29, 11.08320
30, 10.94497
\end{filecontents*}
%end for fig 7

\begin{document}
%\csvautotabular{data2.csv}
\setbeamertemplate{headline}{\leavevmode}
\begin{frame}
\headereven{2}




%%%%%%%%%%%%%%%%%%%%%%%%%%%%%%%%%%%
% P2 Block 1
\begin{textblock}{19.1}(0.25,0.7)\vspace{-11pt}
\setbeamercolor*{block title}{fg=white,bg=Block0}
\setbeamercolor*{block body}{fg=black,bg=Block0!10}
\begin{block}{\Huge Diagramme}


\vspace{0.5cm}
%%%%%%%%%%%%%%%%%%%%%%%%%%%%%%
% 3 Minipages: (19.1-0.5)/3
% Informationen zu Diagrammen s. http://www.bbc.co.uk/education/guides/z7xcwmn/revision/2
% minipage 1
\begin{minipage}[t][6.6cm][t]{6.2cm-3ex}\vspace{-3ex}
%\begin{tikzpicture}
%\node[font={\fontsize{9pt}{11}\selectfont}] at (2,1.35) {Kreis-/Tortendiagramm};
%\hspace{0.6cm}
%\node[anchor= north west, text width=4.75cm, align=left] at (-0.05,1.35) {
%\begin{itemize}\itemsep0pt
%\item Der Vollkreis repr�sentiert alle Daten
%\item  Tortenst�cke  repr�sentieren Anteile
%\item  Fl�chenanteil = Anteil in den Daten 
%\end{itemize}
%};
%\end{tikzpicture}
%\vspace{0.2cm}

\centering
\begin{tikzpicture}[baseline]

\node at (4.75/2,1.7){
\begin{tikzpicture} [baseline, scale=.55]
\pie [rotate = 188] 
{
%7.11/ES(20),
%21.70/DE(61),
%8.54/USA(24),
%28.82/FR(81),
%9.60/GB(27),
%24.29/CA(68)
11.4/DE(29),
8.2/US(21),
8.6/FR(22),
23.9/GB(61),
18.4/JP(47),
19.2/CA(49),
10.2/ES(26)
} 
\end{tikzpicture} 
};


\node[align=center,rounded corners = 3pt, draw, text width=1.4cm,anchor=north] (p2n1) at (4.3,0.4) {61 Mitarbeiter in GB $\approx$ 23.9\%};
\draw[->] (p2n1) to [out=90, in =-110] (4.6,0.7);
\node[anchor=west, align=left] at (0,-0.4) {(Mitarbeiter nach L�ndern)};


\node[font={\fontsize{9pt}{11}\selectfont}] at (4.75/2,4.2) {Kreis-/Tortendiagramm};


\node[anchor= north , text width=4.75cm, minimum height=1cm, align=left,rounded corners = 3pt,draw,fill=Block0!25] at (4.75/2,-0.7) {
\vspace{-1.5ex}
\begin{itemize}\itemsep0pt
\item Kreis = alle Daten (hier $N$=255)
\item Tortenst�cke = Anteile in Prozent 
\item  Prozentanteil x $ 360^\circ $ = Winkel 
\end{itemize}
};
\end{tikzpicture}



\end{minipage}
%
\hspace{0.25cm}\hspace{1.5ex}
% minipage 2
\begin{minipage}[t][6.6cm][t]{6.2cm-3ex}\vspace{-3ex}

\centering
\begin{tikzpicture}
 %%more info on page 112 of pgfplot manual
\begin{axis}[height=3.75cm,width=4.75cm,
ylabel style={overlay},yticklabel style={overlay},
scale only axis, 
axis lines = left,
axis line style = very thick,
axis on top, % damit die Balken nicht �ber die Achse reichen, s. http://tex.stackexchange.com/questions/168712/axis line style = very thick,
xbar, xtick={0,10,20,30,40,50,60, 70},
xmin=0,
xmax=70,
enlarge y limits=.15, % s. http://www.scriptscoop.net/t/99958316e56d/pgfplots-cyrillic-text-in-symbolic-x-coords.html (hier x und y tauschen)
symbolic y coords={% von unten nach oben
	ES, CA, JP,GB, FR, US, DE
	},
ytick=data,
nodes near coords, 
nodes near coords align={horizontal},
bar shift=0pt,
%bar width=3.4mm,
axis background/.style={shade,bottom color=Kasten1!20,top color=Block1!20,shading angle=-45}]
\addplot[fill=dbscan] coordinates {
(29,{DE})	(26,{ES}) (49,{CA}) (47,{JP}) (61,{GB}) (22,{FR}) (21,{US}) 	
};

\end{axis}  


\node[font={\fontsize{9pt}{11}\selectfont}] at (4.75/2,4.2) {Balkendiagramm};
\node[anchor= north , text width=4.75cm, minimum height=1cm, align=left,rounded corners = 3pt,draw,fill=Block0!25] at (4.75/2,-0.7) {
\vspace{-1.5ex}
\begin{itemize}\itemsep0pt
\item Liegendes S�ulendiagramm
\item Merkmalsauspr�gung auf der Y-Achse 
\item  H�ufigkeiten (abs./rel.) auf der X-Achse
\end{itemize}
};

\node[anchor=east,xshift=2mm] at (4.75,-0.5) {Anz.\ Mitarbeiter};
\node[align=center,rounded corners = 3pt, draw, text width=1.5cm,anchor=north] (p2n2) at (3.4,3.4) {61 Mitarbeiter in GB};
\draw[->] (p2n2) to [out=-90, in =90] (61/70*4.75,2.1);
\end{tikzpicture}


\end{minipage}
%
% minipage 3
\hspace{0.25cm}\hspace{1.5ex}
\begin{minipage}[t][6.6cm][t]{6.2cm-3.2ex}\vspace{-3ex}
\centering
\begin{tikzpicture}
\begin{axis}[scale only axis, ylabel style={overlay},yticklabel style={overlay}, 
xlabel style={overlay},xticklabel style={overlay}, 
ybar interval, %ylabel=$\hat f(x)$,%xlabel=K�rpergr��e,
axis lines = left, every axis y label/.style={at=(current axis.above origin),below=1mm,anchor=north west},
every axis x label/.style={at=(current axis.right of origin), right=2mm}, xmin= 155, xmax=195, ymax=0.06, ymin=0, 
every axis y label/.style={at=(current axis.above origin),anchor=south},
height=3.75cm, width=4.75cm,
xtick={160,165,170,175,180,185,190}, xticklabels={160,165,170,175,180,185,190},
ytick={0.01,0.02,0.03,0.04,0.05,0.06}, 
yticklabels={0.01,0.02,0.03,0.04,0.05,0.06}, 
scaled ticks=false, xticklabel style={xshift=-0.3cm},
axis background/.style={shade,bottom color=Kasten1!20,top color=Block2!20,shading angle=-45},
axis line style = very thick,axis on top,
xmajorgrids=false % http://tex.stackexchange.com/questions/248807/draw-grid-lines-for-grouped-bar-chart-diagram
]
\addplot[col_plot2, fill=col_plot2!50, line width=1.5pt] coordinates
	{(160,0.03)(165,0.04)(170,0.05) (175,0.04) (180,0.02)(185,0.02)(190,0)};
%\draw [decorate, decoration={brace,amplitude=1.5pt,raise=1pt}] (150,520) -- (200,520);
\end{axis}


\node[font={\fontsize{9pt}{11}\selectfont}] at (4.75/2,4.2) {Histogramm};
\node[anchor= north , text width=4.75cm, minimum height=1cm, align=left,rounded corners = 3pt,draw,fill=Block0!25] at (4.75/2,-0.7) {
\vspace{-1.5ex}
\begin{itemize}\itemsep0pt
\item In Klassen eingeteilte stetige Daten
\item Rechteckfl�che = relative H�ufigkeit
\item Rechteckh�he = (H�ufigkeits-) Dichte
\end{itemize}
};


\node[anchor=east,xshift=2mm] at (4.75,-0.5) {K�rpergr��e};
\node[align=center,rounded corners = 3pt, draw, text width=1.7cm,anchor=north west] (p2n3) at (0.1,3.7) {Dichte = 0.05};
\draw[->] (p2n3) to [out=-130, in =180] (1.7,5/6*3.75);
\node[align=center,rounded corners = 3pt, draw, text width=1.45cm,anchor=north east] (p2n31) at (4.7,3.7) {H�he x Breite = 0.25, d.h.\ 25\% der Befragten sind zwischen 170 und 175 cm gro�};
\draw[->] (p2n31) to [out=180, in =30] (2.1,2.7);
\end{tikzpicture}
\end{minipage}


%%%%%%%%%%%%%%%%%%%%%%%%%%%%%
% minipages 4-6
% Minipage 4
\begin{minipage}[t][6.6cm][t]{6.2cm-3ex}\vspace{-3ex}
\centering
\begin{tikzpicture}

\begin{axis}[ylabel style={overlay},yticklabel style={overlay}, % s. http://texwelt.de/wissen/fragen/2997/wie-kann-ich-zwei-pgfplots-diagramme-ubereinander-ausrichten
axis lines = middle,
scale only axis,    
axis y line*=none, height=3.75cm,width=4.75cm,
axis x line*=none,
xmin=0,
xmax=5,
xtick={1,2,3,4},
xticklabels={2003,2004,2005,2006},
ymin=-3,ymax=3,
yticklabel={\pgfmathprintnumber\tick\%},axis line style = very thick,
%nodes near coords,
%title=Distribution Function in \%
axis background/.style={shade,bottom color=Kasten1!20,top color=Block1!20,shading angle=-45}]
\addplot[ybar,bar width=6mm,bar shift=-0in,fill=dbscan,draw=black]  coordinates{
(1,1.5)
(4,1.7)};
\addplot [
color=black,
solid, 
forget plot
]
table[row sep=crcr]{
-.33 0\\
12 0\\
};
\addplot[ybar,bar width=6mm,bar shift=0.0in,fill=findOptimalPartition,draw=black] plot coordinates{
(2,-1)
(3,-2.1)};

\node[align=center,anchor=south] at (axis cs: 1,1.5) {1.5\%};
\node[align=center,anchor=north] at (axis cs: 2,-1) {-1\%};
\node[align=center,anchor=north] at (axis cs: 3,-2.1) {-2.1\%};
\node[align=center,anchor=south] at (axis cs: 4,1.7) {1.7\%};



\end{axis}
\node[anchor=east,xshift=2mm] at (4.75,-0.5) {Umsatzwachstum};
\node[align=center,rounded corners = 3pt, draw, text width=1.5cm,anchor=north ] (p2n4) at (4.75/2,3.5) {Im Jahr 2004 sank der Umsatz um 1\%};
\draw[->] (p2n4) to [out=-90, in =90] (2/5*4.75,1.4);



\node[font={\fontsize{9pt}{11}\selectfont}] at (4.75/2,4.2) {S�ulendiagramm};
\node[anchor= north , text width=4.75cm, minimum height=1cm, align=left,rounded corners = 3pt,draw,fill=Block0!25] at (4.75/2,-0.7) {
\vspace{-1.5ex}
\begin{itemize}\itemsep0pt
\item Auspr�gungen auf der X-Achse 
\item Prozente/Anzahl auf der Y-Achse 
\item Breite der S�ulen ohne Bedeutung
\end{itemize}
};

\end{tikzpicture}%

\end{minipage}
%
\hspace{0.25cm}\hspace{1.5ex}
% Minipage 5
\begin{minipage}[t][6.2cm][t]{6.2cm-3ex}\vspace{-3ex}
\centering
\begin{tikzpicture}
\begin{axis}[height=3.75cm,width=4.75cm,
ylabel style={overlay},yticklabel style={overlay},
scale only axis, 
axis lines = left,
axis line style = very thick,
axis on top, % damit die Balken nicht �ber die Achse reichen, s. http://tex.stackexchange.com/questions/168712/axis line style = very thick,
xbar, xtick={0,10,20,30,40,50,60, 70},
xmin=0,
xmax=70,
enlarge y limits=.15, % s. http://www.scriptscoop.net/t/99958316e56d/pgfplots-cyrillic-text-in-symbolic-x-coords.html (hier x und y tauschen)
symbolic y coords={% von unten nach oben
	 ES, CA,JP,GB, FR, US, DE, 
	},
ytick=data,
nodes near coords,
every node near coord/.append style={anchor=east,xshift=0.75ex},
%every node near coord/.append style={xshift=-1.5ex}, % http://tex.stackexchange.com/questions/270721/how-to-correct-nodes-near-coords-position-in-ybar-stacked
%every node near coord/.append style={xshift=-15pt,yshift=-3pt,anchor=east},
%nodes near coords align={horizontal},
bar shift=0pt,
%bar width=3.4mm,
axis background/.style={shade,bottom color=Kasten1!20,top color=Block1!20,shading angle=-45},
legend style={fill=none},
legend pos=south east,
%every axis y label/.style={at=(current axis.above origin),anchor=south},
%every axis x label/.style={at=(current axis.right of origin),anchor=west},
xbar stacked,
%nodes near coords, nodes near coords align={horizontal}, Suche: tikz "xbar stacked" nodes
%legend style={at={(0.8,0.2)},anchor=west, draw=none},
%area legend,
%y=5.77mm,
%enlarge y limits={abs=0.626},
%legend style={at={(1.2,0.5)},legend columns=-1
%},
%
%http://tex.stackexchange.com/questions/125935/customizing-legend-in-pgfplots
% den Parameter #1, musste man aus irgendeinem Grund entfernen...
legend image code/.code={%
                    \draw[draw=none] (0cm,-0.1cm) rectangle (0.3cm,0.1cm);
                },  
legend style={draw=none,at={(1.03,0.05)}, anchor=south east}]
\addplot[fill=dbscan] coordinates {
(29,{CA})  (13,{ES}) (36,{JP}) (56,{GB}) (4,{FR}) (17,{US}) (20,{DE})};
\addplot[fill=red] coordinates {
(20,{CA})  (13,{ES}) (11,{JP}) (5,{GB}) (18,{FR}) (4,{US}) (9,{DE})};
%\legend{XXX, YYY}

\legend{\strut Frauen, \strut M�nner};
%\legend{Far,Near}
%\node[anchor=center] at (rel axis cs: 0.5,0.5) {Hallo};


\end{axis}  
%\node at (1,1) {Hallo};
%\node[anchor=center] at (29/70*4.75,6.5/7*3.75) {Hallo};



\node[font={\fontsize{9pt}{11}\selectfont}] at (4.75/2,4.2) {Gestapeltes Balkendiagramm};
\node[anchor= north , text width=4.75cm, minimum height=1cm, align=left,rounded corners = 3pt,draw,fill=Block0!25] at (4.75/2,-0.7) {
\vspace{-1.5ex}
\begin{itemize}\itemsep0pt
\item Mehrere Einzelwerte nebeneinander  
\item Addieren sich zum Gesamtwert auf
\item Visualisiert Gesamtzusammensetzung
\end{itemize}
};



\node[anchor=east,xshift=2mm] at (4.75,-0.5) {Anz.\ Mitarbeiter};
\node[align=center,rounded corners = 3pt, draw, text width=2.2cm,anchor=north east] (p2n2) at (4.7,3.7) {Insgesamt 61 Mitarbeiter in GB:  56 Frauen und 5 M�nner};
\draw[->] (p2n2) to [out=-90, in =90] (61/70*4.75,2.1);
\end{tikzpicture}


\end{minipage}
%
\hspace{0.25cm}\hspace{1.5ex}
% Minipage 6
\begin{minipage}[t][6.2cm][t]{6.2cm-3.2ex}\vspace{-3ex}
\centering
\begin{tikzpicture}
\begin{axis}[scale only axis,  axis line style = very thick,
ylabel style={overlay},yticklabel style={overlay}, xlabel style={overlay},xticklabel style={overlay}, 
axis lines = left, every axis y label/.style={at=(current axis.above origin),anchor=south},
every axis x label/.style={at=(current axis.right of origin),anchor=west},
%xmin= 0, xmax=8, ymin=0,ymax=1, 
height=3.75cm,width=4.75cm,
xmin= 155, xmax=195, ymax=1.05, ymin=0, 
 xtick={160,165,170,175,180,185,190}, xticklabels={160,,170,,180,,190},
  ytick={0,0.2,0.4,0.6,0.8,1}, 
yticklabels={0,0.2,0.4,0.6,0.8,1}, 
scaled ticks=false, axis background/.style={shade,bottom color=Kasten1!20,top color=Block1!20,shading angle=-45}]
\addplot+[ name path=A, mark size=1pt,mark options={solid},color=col_plot2, line width=1.5pt] coordinates 
	{(150,0)(160,0)(165,0.15)(170,0.35) (175,0.60) (180,0.80)(185,0.90)(190,1)(200,1)};
%ytick={0.2,0.4,0.6,0.8,1.0}, 
%yticklabels={0.2,0.4,0.6,0.8,1.0}, 
%scaled ticks=false,
%xtick={1,2,3,4,5,6,7,8}, xticklabels={160, ,170, ,180, ,190, ,}, 
%axis background/.style={shade,bottom color=Kasten1!20,top color=Block1!20,shading angle=-45},axis line style = very thick,axis on top]
%\addplot[const plot, mark size=1pt, mark options={solid},color=col_plot2, line width=1.5pt]  coordinates
%{(0,0) (1,0.15) (2,0.55) (3,0.6) (4,0.75) (5,0.95) (6,1) (7,1)};
%\draw[->] (axis cs: 4,0.6) -- (axis cs: 0,0.6);
\end{axis}
\draw[dashed,] (4.75/2,0) -- (4.75/2,0.6/1.05*3.75);
\draw[dashed,->] (4.75/2,0.6/1.05*3.75) -- (0,0.6/1.05*3.75);


%adding node of some text
\node[align=center,rounded corners = 3pt, draw, text width=1.5cm,anchor=north east] (p2n6) at (4.7,2.8) {60\% (=0.6) der Personen in der Stichprobe haben eine K�rpergr��e von h�chstens 175 cm};
\draw[->] (p2n6) to [out=-180, in =0] (2.5,0.6/1.05*3.75);
\node[anchor=east,xshift=2mm] at (4.75,-0.5) {K�rpergr��e};



\node[font={\fontsize{9pt}{11}\selectfont}] at (4.75/2,4.2) {Verteilungsfunktion};
\node[anchor= north , text width=4.75cm, minimum height=1cm, align=left,rounded corners = 3pt,draw,fill=Block0!25] at (4.75/2,-0.7) {
\vspace{-1.5ex}
\begin{itemize}\itemsep0pt
\item Y-Achse immer von 0 bis 1 (100\%)
\item   Anteil der Werte in der Stichprobe, die kleiner als ein bestimmter X-Wert sind
\end{itemize}
};


\end{tikzpicture}%
\end{minipage}

%%Minipage - 7
% Minipage 6
\begin{minipage}[t][6.2cm][t]{6.2cm-3ex}\vspace{-3ex}
\centering
\begin{tikzpicture}
\pgfplotsset{every axis/.append style={
extra description/.code={
%\node[font={\fontsize{9pt}{11}\selectfont}] at (0.5,1.2) { Figure 9};
%\node at (.1,1.15) {\textbullet xyz };
%\node at (.1,1.1) {\textbullet xyz };
}}%, every non boxed x axis/.append style={x axis line style==>} 
} %%more info on page 112 of pgfplot manual
\begin{axis}[scale only axis, ylabel style={overlay},yticklabel style={overlay},
y label style={at={(axis description cs:0.18,.46)},rotate=0,anchor=south},
x label style={at={(axis description cs:.46,.06)}},
    ylabel={aufsummierter Anteil Einkommen},
    xlabel={aufsummierter Anteil Bev�lkerung},
height=3.75cm,width=4.75cm,
%xlabel=Körpergrö�?e $X$, ylabel=Gewicht $Y$,
axis lines = left, 
%every axis y label/.style={at=(current axis.above origin),anchor=south},
%  every axis x label/.style={at=(curreant axis.right of origin),anchor=west}, 
%    x label style={at={(axis description cs:0.5,-0.01)},anchor=north},
%    y label style={at={(axis description cs:-0,.5)},rotate=360,anchor=south},
  xmin= 0, xmax=100, ymin = 0, ymax=100, 
 xtick={0,25,50,75,100}, xticklabels={0,25,50,75,100},
  ytick={0,25,50,75,100}, 
yticklabels={0,25,50,75,100}, axis background/.style={shade,bottom color=Kasten1!20,top color=Block1!20,shading angle=-45},
scaled ticks=false,]
	%%regression line code, equation y=1.544x-204.77
%%	\addplot[no markers,blue,samples=10,domain=0:100] {1*(x)-(0)};   
%%	\addplot[no markers,blue,samples=10,domain=0:50] {.5*(x)-(0)};
\draw (0,0) -- (100,100);
%\draw (0,0) -- (10,2) -- (20,6) -- (30,10) -- (40,18) -- (50,25) -- (60,30) -- (70, 40) -- (80, 55) -- (90, 70) -- (100,100);
\addplot[black,fill=black!1] table [x=x
, y=y, col sep=comma] {data3.csv};

\fill [fill=orange!25] (0,0) -- (100,100) -- (10,2) -- (0,0) -- cycle;
\fill [fill=orange!25] (10,2) -- (100,100) -- (20,6) -- (10,2) -- cycle;
\fill [fill=orange!25] (20,6) -- (100,100) -- (30,10) -- (20,6) -- cycle;
\fill [fill=orange!25] (30,10) -- (100,100) -- (40,18) -- (30,10) -- cycle;
\fill [fill=orange!25] (40,18) -- (100,100) -- (50,25) -- (40,18) -- cycle;
\fill [fill=orange!25] (50,25) -- (100,100) -- (60,30) -- (50,25) -- cycle;
\fill [fill=orange!25] (60,30) -- (100,100) -- (70,40) -- (60,30) -- cycle;
\fill [fill=orange!25] (70,40) -- (100,100) -- (80,55) -- (70,40) -- cycle;
\fill [fill=orange!25] (80,55) -- (100,100) -- (90,70) -- (80,55) -- cycle;
\fill [fill=orange!25] (90,70) -- (100,100) -- (100,100) -- (90,70) -- cycle;

\filldraw (0,0) circle (1.2pt);
\filldraw (10,2) circle (1.2pt);
\filldraw (20,6) circle (1.2pt);
\filldraw (30,10) circle (1.2pt);
\filldraw (40,18) circle (1.2pt);
\filldraw (50,25) circle (1.2pt);
\filldraw (60,30) circle (1.2pt);
\filldraw (70,40) circle (1.2pt);
\filldraw (80,55) circle (1.2pt);
\filldraw (90,70) circle (1.2pt);
\filldraw (100,100) circle (1.2pt);

\node[font={\fontsize{4pt}{11}\selectfont}, align=center, text width=1cm,anchor=north] (f9n1) at (15,80) {Gleichverteilung};
\draw[->] (f9n1) to [out=270, in =180] (55,55);

\node[font={\fontsize{4pt}{11}\selectfont}, align=center, text width=1cm,anchor=north] (f9n1) at (80,25) {Ungleichheit};
\draw[->] (f9n1) to [out=90, in =0] (60,37);

\end{axis}

%adding node of some text

\node[font={\fontsize{9pt}{11}\selectfont}] at (4.75/2,4.2) {Figure 7};
\node[anchor= north , text width=4.75cm, minimum height=1cm, align=left,rounded corners = 3pt,draw,fill=Block0!25] at (4.75/2,-0.7) {
\vspace{-1.5ex}
\begin{itemize}\itemsep0pt
\item xyz
\item   xyz
\end{itemize}
};
\end{tikzpicture}%


\end{minipage}
%
\hspace{0.25cm}\hspace{3.4ex}
% Minipage 8
\begin{minipage}[t][6.2cm][t]{6.2cm-3.2ex}\vspace{-3ex}
\centering
\begin{tikzpicture}
\pgfplotsset{every tick label/.append style={font=\tiny},every axis/.append style={
extra description/.code={
%\node[font={\fontsize{9pt}{11}\selectfont}] at (0.5,1.2) { Figure 9};
%\node at (.1,1.15) {\textbullet xyz };
%\node at (.1,1.1) {\textbullet xyz };
}}%, every non boxed x axis/.append style={x axis line style==>} 
} %%more info on page 112 of pgfplot manual
\begin{axis}[ymajorgrids, scale only axis, ylabel style={overlay},yticklabel style={overlay},
height=3.75cm,width=4.75cm,
%xlabel=Körpergröße $X$, ylabel=Gewicht $Y$,
axis lines = left, 
y label style={at={(axis description cs:0.18,.46)},rotate=0,anchor=south},
x label style={at={(axis description cs:.46,.06)}},
    ylabel={Public Dept (\$U.S Billions)},
    xlabel={Public Debt \% GDP},
%every axis y label/.style={at=(current axis.above origin),anchor=south},
%  every axis x label/.style={at=(curreant axis.right of origin),anchor=west}, 
%    x label style={at={(axis description cs:0.5,-0.01)},anchor=north},
%    y label style={at={(axis description cs:-0,.5)},rotate=360,anchor=south},
   ymin= 0, ymax=350, xmin = 0, xmax=120, 
ytick={0,50,100,150,200,250,300,350}, yticklabels={0,500,1000,1500,2000,2500,3000,3500},
  xtick=    {0, 10,20,30,40,50,60, 70,80,   90,100,110,120}, 
xticklabels={40,50,60,70,80,90,100,110,120,130,140,150,160},
axis background/.style={shade,bottom color=Kasten1!20,top color=Block1!20,shading angle=-45},
scaled ticks=false,]
%$-20
%\filldraw[color=red!5, fill=red!5, very thin](0,100) circle (.3cm);
  \draw[fill=portugal] (53,21.4) circle [radius=0.15cm]; %Portugal
  \draw[fill=ireland] (55,19.4) circle [radius=0.15cm]; %Ireland
%  \draw[fill=iceland] (83,.25) circle [radius=0.05cm]; %Iceland
  \draw[fill=italy] (79,244.5) circle [radius=.48cm]; %Greece
  \draw[fill=greece] (103,43.6) circle [radius=0.25cm]; %Italy
  \draw[fill=germany] (43,275.2) circle [radius=0.415cm]; %Germany
  \draw[fill=spain] (20,82.1) circle [radius=0.3cm]; %spain
  \draw[fill=france] (43,214.4) circle [radius=0.415cm]; %France
  \draw[fill=uk] (36,170.8) circle [radius=0.38cm]; %UK
%
\end{axis}


%adding node of some text
\node[font={\fontsize{4pt}{4}\selectfont},align=center, text width=1.5cm,anchor=north east] (p2n611) at (3.98,2.95) {Italy \\ 119\% \\ \$2,445};
\node[font={\fontsize{4pt}{4}\selectfont},align=center, text width=1.5cm,anchor=north east] (p2n611) at (2.55,3.22) {Germany \\ 83\% \\ \$2,752};
\node[font={\fontsize{4pt}{4}\selectfont},align=center, text width=1.5cm,anchor=north east] (p2n611) at (2.55,2.67) {France \\ 83\% \\ \$2,114};
\node[font={\fontsize{4pt}{4}\selectfont},align=center, text width=1.5cm,anchor=north east] (p2n611) at (2.27,2.12) {UK \\ 76\% \\ \$1,708};
\node[font={\fontsize{4pt}{4}\selectfont},align=center, text width=1.5cm,anchor=north east] (p2n611) at (1.63,1.15) {Spain \\ 60\% \\ \$821};
%\draw (3.8,.6) -- (3.6,.8);
\draw[->]  (4.42,.8) -- (4.3,.7);
\node[font={\fontsize{4pt}{4}\selectfont},align=center, text width=1.5cm,anchor=north east] (p2n611) at (5.34,1.3) {Greece \\ 143\% \\ \$436};
\draw[->]  (3.4,.23)--(3.32,.1);
\node[font={\fontsize{4pt}{4}\selectfont},align=center, text width=1.5cm,anchor=north east] (p2n611) at (4.35,.8) {Iceland \\ 123\% \\ \$15};
\draw[->] (2.5,.44)--(2.35,.3) ;
\node[font={\fontsize{4pt}{4}\selectfont},align=center, text width=1.5cm,anchor=north east] (p2n611) at (3.58,.9) {Ireland \\ 95\% \\ \$194};
\draw[->]  (1.78,.27)--(1.9,.27);
\node[font={\fontsize{4pt}{4}\selectfont},align=center, text width=1.5cm,anchor=north east] (p2n611) at (2.35,.63) {Portugal \\ 93\% \\ \$214};
\draw[fill=iceland] (3.26,.02) circle [radius=0.05cm]; %Iceland
\node[font={\fontsize{9pt}{11}\selectfont}] at (4.75/2,4.2) {Figure 8};
\node[anchor= north , text width=4.75cm, minimum height=1cm, align=left,rounded corners = 3pt,draw,fill=Block0!25] at (4.75/2,-0.7) {
\vspace{-1.5ex}
\begin{itemize}\itemsep0pt
\item xyz
\item   xyz
\end{itemize}
};

\end{tikzpicture}%
\end{minipage}
%
\hspace{0.1cm}\hspace{-.4ex}
% Minipage 8
\begin{minipage}[t][6.2cm][t]{6.2cm-3.2ex}\vspace{-3ex}
\centering
\begin{tikzpicture}
\pgfplotsset{every x tick label/.append style={font=\tiny}}
\begin{axis}[scale only axis,name=popaxis,axis x line=left, 
 height=3.75cm, width=2.085cm,
    legend pos=south east,
    xbar stacked,
%legend style={at={(0.8,0.2)},anchor=west, draw=none},
ytick=data,
legend style={fill=none},
legend pos=south east,
axis line style={-},
ytick pos=right,    ytick style={/pgfplots/on layer=axis foreground, very thin, black},
xtick style={/pgfplots/on layer=axis foreground, very thin, black},   
    yticklabel style={/pgfplots/on layer=axis foreground},
    axis y line*=none,
    axis x line*=bottom,
    tick label style={font=\footnotesize},
    label style={font=\footnotesize},
    xtick={0,100,200,300,400,500,600,700,750},
    xticklabels={0,100,200,300,400,500,600,700,},
%    width=.9\textwidth,
 ytick pos=left,
 ytick align=center,
    bar width=0.2mm, %change here
 %   xlabel={Stacked Bar chart},
  yticklabels={
  ,,,,5,,,,,10,
  ,,,,15,,,,,20,
  ,,,,25,,,,,30,
  ,,,,35,,,,,40,
  ,,,,45,,,,,50,
  ,,,,55,,,,,60,
  ,,,,65,,,,,70,
  ,,,,75,,,,,80,
  ,,,,85,,,,,90,
  ,,,,95,,,,,100,
	},
      xmin=0,
    xmax=800,ytick align=center,
    area legend,
%    y=8mm,
    enlarge y limits={abs=2.3},    ytick align=center,
    %%alignment
    %%Algnment ends here
%    xshift=0.42239cm,
    scale only axis, 
    set layers,
%ytick align=center,
    clip=false,
    axis on top,axis background/.style={shade,bottom color=Kasten1!20,top color=Block1!20,shading angle=-45}
]
%female section

\addplot[gray1,fill=gray1] table [x=fem_grey1
, y=y, col sep=comma] {data2.csv};
\addplot[green1,fill=green1] table [x=fem_green1
, y=y, col sep=comma] {data2.csv};
\addplot[green2,fill=green2] table [x=fem_green2
, y=y, col sep=comma] {data2.csv};
\addplot[orange1,fill=orange1] table [x=fem_orange1
, y=y, col sep=comma] {data2.csv};
\addplot[orange2,fill=orange2] table [x=fem_orange2
, y=y, col sep=comma] {data2.csv};

%\legend{XXX, YYY}
%legend image code/.code={%
 %                   \draw[draw=none] (0cm,-0.1cm) rectangle (0.1cm,0.1cm);
%                },  
%legend style={draw=none,at={(1.03,0.05)}, anchor=south east}]
%\legend{\strut Frauen, \strut M�nner, \strut Umer};

\end{axis}  
%MALE sec start
\begin{axis}[at={(popaxis.south west)},anchor=south east, 
xtick style={/pgfplots/on layer=axis foreground, very thin, black},   
  height=3.75cm, width=2.085cm,
 axis x line=left,
    axis y line=left,    ytick align=center,
    ytick style={/pgfplots/on layer=axis foreground, very thin, black},
    yticklabel style={/pgfplots/on layer=axis foreground},
    scale only axis,scale only axis,x dir=reverse,xshift=-.58cm,
    legend pos=south east,
    xbar stacked,
%legend style={at={(0.8,0.2)},anchor=west, draw=none},
ytick=data,
    axis y line*=none,    ytick align=center,
    axis x line*=bottom,
    tick label style={font=\footnotesize},
    label style={font=\footnotesize},
    xtick={0,100,200,300,400,500,600,700,750},
    xticklabels={0,100,200,300,400,500,600,700,},
%    width=.9\textwidth,
    bar width=.2mm,
%    xlabel={Stacked Bar chart},
  yticklabels={
  ,,,, ,,,,,,
  ,,,, ,,,,,,
  ,,,, ,,,,,,
  ,,,, ,,,,,,
  ,,,, ,,,,,,
  ,,,, ,,,,,,
  ,,,, ,,,,,,
  ,,,, ,,,,,,
  ,,,, ,,,,,,
  ,,,,  ,,,,,,
	},%21
      xmin=0,
    xmax=800,
        area legend,    ytick pos=right,ytick align=center,
    %y=8mm,
    enlarge y limits={abs=2.3},
    %yshift=2,
    %%alignment
    %%Algnment ends here
%    xshift=0.42239cm,
    scale only axis, 
    set layers,    %enlarge x limits = {value=0.15,upper},
%ytick align=center,
    clip=false,
    axis on top,    axis line style={-},axis background/.style={shade,bottom color=Kasten1!20,top color=Block1!20,shading angle=-45}
]

\addplot[gray1,fill=gray1] table [x=fem_grey1
, y=y, col sep=comma] {data2.csv};
\addplot[gray2,fill=gray2] table [x=male_grey2
, y=y, col sep=comma] {data2.csv};
\addplot[green1,fill=green1] table [x=fem_green1
, y=y, col sep=comma] {data2.csv};
\addplot[green2,fill=green2] table [x=male_green2
, y=y, col sep=comma] {data2.csv};

\addplot[orange1,fill=orange1] table [x=fem_orange1
, y=y, col sep=comma] {data2.csv};
\end{axis}  
\fill[draw=orange1,color=orange1] (-2.60,3.5) rectangle (-2.55,3.7);
\fill[draw=green1,color=green1] (-2.60,3.3) rectangle (-2.55,3.5);
\fill[draw=gray1,color=gray1] (-2.60,3.1) rectangle (-2.55,3.3);
\node[font={\fontsize{4pt}{4}\selectfont},anchor=east] at (-1.4,3.62) {65 and above};
\node[font={\fontsize{4pt}{4}\selectfont},anchor=east] at (-1.86,3.4) {20 - 64};
\node[font={\fontsize{4pt}{4}\selectfont},anchor=east] at (-2.08,3.2) {\textless20};
\node[font={\fontsize{5pt}{5}\selectfont},anchor=east] at (2,-0.4) {Women(thousand)};
\node[font={\fontsize{5pt}{5}\selectfont},anchor=east] at (-0.8,-0.4) {Men(thousand)};
\node[font={\fontsize{9pt}{11}\selectfont}] at (-.2,4.2) {Figure 9};
\node[anchor= north , text width=4.75cm, minimum height=1cm, align=left,rounded corners = 3pt,draw,fill=Block0!25] at (-0.4,-0.7) {
\vspace{-1.5ex}
\begin{itemize}\itemsep0pt
\item xyz
\item   xyz
\end{itemize}
};

\end{tikzpicture}%
\end{minipage}



\end{block}
\end{textblock}

\end{frame}
\end{document}
